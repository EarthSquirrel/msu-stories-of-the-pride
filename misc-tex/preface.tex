% preface.tex stores basic background and motivational information about the 
% project. It should appear in the preface of the book.
I was inspired when we went on a fieldtrip for the Queer Straight Alliance 
(QSA) to the library archives. We got to look at QSA’s history, one of the 
oldest clubs on campus \todo{Clarify is it the oldest or just one of the 
oldest?}. I wanted to help document this history as it happens. The purpose of 
this document is to capture stories of queer individuals at Montana State 
University (MSU). It is meant to serve as a time capsule of sorts for LGBTQ+ 
students. The stories are not required to meet any standards; there are no 
assigned topics nor length limit. Participants can write about what is 
meaningful to them and their personal experiences. If things change and people
have more to say, they can add multiple stories over their time at MSU. I hope 
this document can become a collaboration between many groups on campus and 
lives on long past me.


The information for each story (only what the writer is comfortable sharing) is 
a combination of:
\begin{itemize}
    \item Name
    \item Pronouns
    \item Year in school
    \item Major
    \item How the writer identifes
    \item Story
    \item Title of story
\end{itemize}
Some potential topic ideas could be:
\begin{itemize}
    \item What are your experiences as a member of the MSU LGBTQ+ community?
    \item When did you discover your identity and how did you feel about it?
    \item What is your coming out story?
    \item What advice do you have to younger or less experienced LGBTQ+ 
        individuals?
    \item What’s a funny/sad/happy/random story that involves your LGBTQ+ 
        identity?
    \item What are some LGBTQ+ issues in the world today and how do they affect
        you?
    \item (Current Event!) If you were forced to go back home due to 
        coronavirus, how did that situation play out?
\end{itemize}

The current event topic was suggested by Anne. I really like it and think it 
would offer an interesting take on the current times we live in from a queer 
point of view. It would make unique stories for future generations to look at. 
My heart goes out to those who had to return home and be once again closeted or 
be unaccepted. 
