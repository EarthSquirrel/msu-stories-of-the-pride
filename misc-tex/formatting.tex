% Test if something is empty, if so do #2 else do #3
\def \ifempty#1{\def\temp{#1} \ifx\temp\empty }
\newcommand{\testEmpty}[3]{\ifempty{#1} #2 \else #3 \fi}

% Make the command for organizing stories
% \story{name}{title}{story}{pronouns}{major}{year}

\newcommand{\story}[6]{
    \section*{\uppercase{\textbf{#1}}: #2}
    #3 \\
    \textit{\testEmpty{#4}{}{-#4}} \\
    % If major is there add , only if year in school is also there
    \textit{\testEmpty{#5}{}{-#5\testEmpty{#6}{}{, #6}}}
}




% Make the command for organizing stories

% Test if something is empty, if so do #2 else do #3
%\def \ifempty#1{\def\temp{#1} \ifx\temp\empty }
%\newcommand{\testEmpty}[3]{\ifempty{#1} #2 \else #3 \fi}

% Used in msu-2019-2020
% \story{name}{title}{story}{pronouns}{major}{year}{flag commands}
% WHAT IS THE DIFFERENCE BETWEEN THIS AND STORY????
\newcommand{\storytwo}[7]{    
    % #1 Name, #2 Story title
    \section*{\uppercase{\textbf{#1}}: #2}
    \begin{centering} 
    % Only do a new line if there were flags to write out
    #7 \testEmpty{#7}{}{\\}
    \end{centering}

    #3 %\\  % Story

    % Why does surrounding \testEmtpy in \textit indent things?
    % Test if #4 is empty, if so, leave blank, if not format pronouns
    \textit{\testEmpty{#4}{}{-#4}}
    
    % #5 Major, #6 Year in school
    % If major is there add , only if year in school is also there
    \textit{\testEmpty{#5}{}{-#5\testEmpty{#6}{}{, #6}}}
}

% \story{name}{title}{story}{pronouns}{major}{year}
\newcommand{\storyonly}[6]{    
    % #1 Name, #2 Story title
    \section*{\uppercase{\textbf{#1}}: #2}
    
    #3 %\\  % Story

    % Why does surrounding \testEmtpy in \textit indent things?
    % Test if #4 is empty, if so, leave blank, if not format pronouns
    \textit{\testEmpty{#4}{}{-#4}}
    
    % #5 Major, #6 Year in school
    % If major is there add , only if year in school is also there
    \textit{\testEmpty{#5}{}{-#5\testEmpty{#6}{}{, #6}}}
}

